% Generated by Sphinx.
\def\sphinxdocclass{report}
\documentclass[letterpaper,10pt,english]{sphinxmanual}
\usepackage[utf8]{inputenc}
\DeclareUnicodeCharacter{00A0}{\nobreakspace}
\usepackage{cmap}
\usepackage[T1]{fontenc}
\usepackage{babel}
\usepackage{times}
\usepackage[Bjarne]{fncychap}
\usepackage{longtable}
\usepackage{sphinx}
\usepackage{multirow}


\title{Enhanced Sampling Toolkit Documentation}
\date{April 07, 2015}
\release{1.0}
\author{Jeremy Tempkin}
\newcommand{\sphinxlogo}{}
\renewcommand{\releasename}{Release}
\makeindex

\makeatletter
\def\PYG@reset{\let\PYG@it=\relax \let\PYG@bf=\relax%
    \let\PYG@ul=\relax \let\PYG@tc=\relax%
    \let\PYG@bc=\relax \let\PYG@ff=\relax}
\def\PYG@tok#1{\csname PYG@tok@#1\endcsname}
\def\PYG@toks#1+{\ifx\relax#1\empty\else%
    \PYG@tok{#1}\expandafter\PYG@toks\fi}
\def\PYG@do#1{\PYG@bc{\PYG@tc{\PYG@ul{%
    \PYG@it{\PYG@bf{\PYG@ff{#1}}}}}}}
\def\PYG#1#2{\PYG@reset\PYG@toks#1+\relax+\PYG@do{#2}}

\expandafter\def\csname PYG@tok@gd\endcsname{\def\PYG@tc##1{\textcolor[rgb]{0.63,0.00,0.00}{##1}}}
\expandafter\def\csname PYG@tok@gu\endcsname{\let\PYG@bf=\textbf\def\PYG@tc##1{\textcolor[rgb]{0.50,0.00,0.50}{##1}}}
\expandafter\def\csname PYG@tok@gt\endcsname{\def\PYG@tc##1{\textcolor[rgb]{0.00,0.27,0.87}{##1}}}
\expandafter\def\csname PYG@tok@gs\endcsname{\let\PYG@bf=\textbf}
\expandafter\def\csname PYG@tok@gr\endcsname{\def\PYG@tc##1{\textcolor[rgb]{1.00,0.00,0.00}{##1}}}
\expandafter\def\csname PYG@tok@cm\endcsname{\let\PYG@it=\textit\def\PYG@tc##1{\textcolor[rgb]{0.25,0.50,0.56}{##1}}}
\expandafter\def\csname PYG@tok@vg\endcsname{\def\PYG@tc##1{\textcolor[rgb]{0.73,0.38,0.84}{##1}}}
\expandafter\def\csname PYG@tok@m\endcsname{\def\PYG@tc##1{\textcolor[rgb]{0.13,0.50,0.31}{##1}}}
\expandafter\def\csname PYG@tok@mh\endcsname{\def\PYG@tc##1{\textcolor[rgb]{0.13,0.50,0.31}{##1}}}
\expandafter\def\csname PYG@tok@cs\endcsname{\def\PYG@tc##1{\textcolor[rgb]{0.25,0.50,0.56}{##1}}\def\PYG@bc##1{\setlength{\fboxsep}{0pt}\colorbox[rgb]{1.00,0.94,0.94}{\strut ##1}}}
\expandafter\def\csname PYG@tok@ge\endcsname{\let\PYG@it=\textit}
\expandafter\def\csname PYG@tok@vc\endcsname{\def\PYG@tc##1{\textcolor[rgb]{0.73,0.38,0.84}{##1}}}
\expandafter\def\csname PYG@tok@il\endcsname{\def\PYG@tc##1{\textcolor[rgb]{0.13,0.50,0.31}{##1}}}
\expandafter\def\csname PYG@tok@go\endcsname{\def\PYG@tc##1{\textcolor[rgb]{0.20,0.20,0.20}{##1}}}
\expandafter\def\csname PYG@tok@cp\endcsname{\def\PYG@tc##1{\textcolor[rgb]{0.00,0.44,0.13}{##1}}}
\expandafter\def\csname PYG@tok@gi\endcsname{\def\PYG@tc##1{\textcolor[rgb]{0.00,0.63,0.00}{##1}}}
\expandafter\def\csname PYG@tok@gh\endcsname{\let\PYG@bf=\textbf\def\PYG@tc##1{\textcolor[rgb]{0.00,0.00,0.50}{##1}}}
\expandafter\def\csname PYG@tok@ni\endcsname{\let\PYG@bf=\textbf\def\PYG@tc##1{\textcolor[rgb]{0.84,0.33,0.22}{##1}}}
\expandafter\def\csname PYG@tok@nl\endcsname{\let\PYG@bf=\textbf\def\PYG@tc##1{\textcolor[rgb]{0.00,0.13,0.44}{##1}}}
\expandafter\def\csname PYG@tok@nn\endcsname{\let\PYG@bf=\textbf\def\PYG@tc##1{\textcolor[rgb]{0.05,0.52,0.71}{##1}}}
\expandafter\def\csname PYG@tok@no\endcsname{\def\PYG@tc##1{\textcolor[rgb]{0.38,0.68,0.84}{##1}}}
\expandafter\def\csname PYG@tok@na\endcsname{\def\PYG@tc##1{\textcolor[rgb]{0.25,0.44,0.63}{##1}}}
\expandafter\def\csname PYG@tok@nb\endcsname{\def\PYG@tc##1{\textcolor[rgb]{0.00,0.44,0.13}{##1}}}
\expandafter\def\csname PYG@tok@nc\endcsname{\let\PYG@bf=\textbf\def\PYG@tc##1{\textcolor[rgb]{0.05,0.52,0.71}{##1}}}
\expandafter\def\csname PYG@tok@nd\endcsname{\let\PYG@bf=\textbf\def\PYG@tc##1{\textcolor[rgb]{0.33,0.33,0.33}{##1}}}
\expandafter\def\csname PYG@tok@ne\endcsname{\def\PYG@tc##1{\textcolor[rgb]{0.00,0.44,0.13}{##1}}}
\expandafter\def\csname PYG@tok@nf\endcsname{\def\PYG@tc##1{\textcolor[rgb]{0.02,0.16,0.49}{##1}}}
\expandafter\def\csname PYG@tok@si\endcsname{\let\PYG@it=\textit\def\PYG@tc##1{\textcolor[rgb]{0.44,0.63,0.82}{##1}}}
\expandafter\def\csname PYG@tok@s2\endcsname{\def\PYG@tc##1{\textcolor[rgb]{0.25,0.44,0.63}{##1}}}
\expandafter\def\csname PYG@tok@vi\endcsname{\def\PYG@tc##1{\textcolor[rgb]{0.73,0.38,0.84}{##1}}}
\expandafter\def\csname PYG@tok@nt\endcsname{\let\PYG@bf=\textbf\def\PYG@tc##1{\textcolor[rgb]{0.02,0.16,0.45}{##1}}}
\expandafter\def\csname PYG@tok@nv\endcsname{\def\PYG@tc##1{\textcolor[rgb]{0.73,0.38,0.84}{##1}}}
\expandafter\def\csname PYG@tok@s1\endcsname{\def\PYG@tc##1{\textcolor[rgb]{0.25,0.44,0.63}{##1}}}
\expandafter\def\csname PYG@tok@gp\endcsname{\let\PYG@bf=\textbf\def\PYG@tc##1{\textcolor[rgb]{0.78,0.36,0.04}{##1}}}
\expandafter\def\csname PYG@tok@sh\endcsname{\def\PYG@tc##1{\textcolor[rgb]{0.25,0.44,0.63}{##1}}}
\expandafter\def\csname PYG@tok@ow\endcsname{\let\PYG@bf=\textbf\def\PYG@tc##1{\textcolor[rgb]{0.00,0.44,0.13}{##1}}}
\expandafter\def\csname PYG@tok@sx\endcsname{\def\PYG@tc##1{\textcolor[rgb]{0.78,0.36,0.04}{##1}}}
\expandafter\def\csname PYG@tok@bp\endcsname{\def\PYG@tc##1{\textcolor[rgb]{0.00,0.44,0.13}{##1}}}
\expandafter\def\csname PYG@tok@c1\endcsname{\let\PYG@it=\textit\def\PYG@tc##1{\textcolor[rgb]{0.25,0.50,0.56}{##1}}}
\expandafter\def\csname PYG@tok@kc\endcsname{\let\PYG@bf=\textbf\def\PYG@tc##1{\textcolor[rgb]{0.00,0.44,0.13}{##1}}}
\expandafter\def\csname PYG@tok@c\endcsname{\let\PYG@it=\textit\def\PYG@tc##1{\textcolor[rgb]{0.25,0.50,0.56}{##1}}}
\expandafter\def\csname PYG@tok@mf\endcsname{\def\PYG@tc##1{\textcolor[rgb]{0.13,0.50,0.31}{##1}}}
\expandafter\def\csname PYG@tok@err\endcsname{\def\PYG@bc##1{\setlength{\fboxsep}{0pt}\fcolorbox[rgb]{1.00,0.00,0.00}{1,1,1}{\strut ##1}}}
\expandafter\def\csname PYG@tok@mb\endcsname{\def\PYG@tc##1{\textcolor[rgb]{0.13,0.50,0.31}{##1}}}
\expandafter\def\csname PYG@tok@ss\endcsname{\def\PYG@tc##1{\textcolor[rgb]{0.32,0.47,0.09}{##1}}}
\expandafter\def\csname PYG@tok@sr\endcsname{\def\PYG@tc##1{\textcolor[rgb]{0.14,0.33,0.53}{##1}}}
\expandafter\def\csname PYG@tok@mo\endcsname{\def\PYG@tc##1{\textcolor[rgb]{0.13,0.50,0.31}{##1}}}
\expandafter\def\csname PYG@tok@kd\endcsname{\let\PYG@bf=\textbf\def\PYG@tc##1{\textcolor[rgb]{0.00,0.44,0.13}{##1}}}
\expandafter\def\csname PYG@tok@mi\endcsname{\def\PYG@tc##1{\textcolor[rgb]{0.13,0.50,0.31}{##1}}}
\expandafter\def\csname PYG@tok@kn\endcsname{\let\PYG@bf=\textbf\def\PYG@tc##1{\textcolor[rgb]{0.00,0.44,0.13}{##1}}}
\expandafter\def\csname PYG@tok@o\endcsname{\def\PYG@tc##1{\textcolor[rgb]{0.40,0.40,0.40}{##1}}}
\expandafter\def\csname PYG@tok@kr\endcsname{\let\PYG@bf=\textbf\def\PYG@tc##1{\textcolor[rgb]{0.00,0.44,0.13}{##1}}}
\expandafter\def\csname PYG@tok@s\endcsname{\def\PYG@tc##1{\textcolor[rgb]{0.25,0.44,0.63}{##1}}}
\expandafter\def\csname PYG@tok@kp\endcsname{\def\PYG@tc##1{\textcolor[rgb]{0.00,0.44,0.13}{##1}}}
\expandafter\def\csname PYG@tok@w\endcsname{\def\PYG@tc##1{\textcolor[rgb]{0.73,0.73,0.73}{##1}}}
\expandafter\def\csname PYG@tok@kt\endcsname{\def\PYG@tc##1{\textcolor[rgb]{0.56,0.13,0.00}{##1}}}
\expandafter\def\csname PYG@tok@sc\endcsname{\def\PYG@tc##1{\textcolor[rgb]{0.25,0.44,0.63}{##1}}}
\expandafter\def\csname PYG@tok@sb\endcsname{\def\PYG@tc##1{\textcolor[rgb]{0.25,0.44,0.63}{##1}}}
\expandafter\def\csname PYG@tok@k\endcsname{\let\PYG@bf=\textbf\def\PYG@tc##1{\textcolor[rgb]{0.00,0.44,0.13}{##1}}}
\expandafter\def\csname PYG@tok@se\endcsname{\let\PYG@bf=\textbf\def\PYG@tc##1{\textcolor[rgb]{0.25,0.44,0.63}{##1}}}
\expandafter\def\csname PYG@tok@sd\endcsname{\let\PYG@it=\textit\def\PYG@tc##1{\textcolor[rgb]{0.25,0.44,0.63}{##1}}}

\def\PYGZbs{\char`\\}
\def\PYGZus{\char`\_}
\def\PYGZob{\char`\{}
\def\PYGZcb{\char`\}}
\def\PYGZca{\char`\^}
\def\PYGZam{\char`\&}
\def\PYGZlt{\char`\<}
\def\PYGZgt{\char`\>}
\def\PYGZsh{\char`\#}
\def\PYGZpc{\char`\%}
\def\PYGZdl{\char`\$}
\def\PYGZhy{\char`\-}
\def\PYGZsq{\char`\'}
\def\PYGZdq{\char`\"}
\def\PYGZti{\char`\~}
% for compatibility with earlier versions
\def\PYGZat{@}
\def\PYGZlb{[}
\def\PYGZrb{]}
\makeatother

\renewcommand\PYGZsq{\textquotesingle}

\begin{document}

\maketitle
\tableofcontents
\phantomsection\label{index::doc}


Contents:


\chapter{Walker API}
\label{src/src.doc:welcome-to-enhanced-sampling-toolkit-s-documentation}\label{src/src.doc:module-walker}\label{src/src.doc::doc}\label{src/src.doc:walker-api}\index{walker (module)}
Created on Fri Jun  6 15:16:18 2014

NOTE: we may want to use abstract properties in the future.

@author: jeremytempkin
\index{velocityWalker (class in walker)}

\begin{fulllineitems}
\phantomsection\label{src/src.doc:walker.velocityWalker}\pysigline{\strong{class }\code{walker.}\bfcode{velocityWalker}}
This is an abstract class for dynamics which have a velocity component, such as 
a protein under Langevin dynamics.  It extends the walker abstract class by
adding additional abstract methods that are necessary for a walker that has
a velocity.
\index{drawVel() (walker.velocityWalker method)}

\begin{fulllineitems}
\phantomsection\label{src/src.doc:walker.velocityWalker.drawVel}\pysiglinewithargsret{\bfcode{drawVel}}{}{}
Draws a new value of the velocities for the walker.

\end{fulllineitems}

\index{reverseVel() (walker.velocityWalker method)}

\begin{fulllineitems}
\phantomsection\label{src/src.doc:walker.velocityWalker.reverseVel}\pysiglinewithargsret{\bfcode{reverseVel}}{}{}
This function reverses the velocity of the walker.

\end{fulllineitems}


\end{fulllineitems}

\index{walker (class in walker)}

\begin{fulllineitems}
\phantomsection\label{src/src.doc:walker.walker}\pysigline{\strong{class }\code{walker.}\bfcode{walker}}
Defines an abstract walker object.  It functions as an interface, providing methods that other
walkers should implement.
\index{close() (walker.walker method)}

\begin{fulllineitems}
\phantomsection\label{src/src.doc:walker.walker.close}\pysiglinewithargsret{\bfcode{close}}{}{}
Destroys the walker, and does some housecleaning.

\end{fulllineitems}

\index{equilibrate() (walker.walker method)}

\begin{fulllineitems}
\phantomsection\label{src/src.doc:walker.walker.equilibrate}\pysiglinewithargsret{\bfcode{equilibrate}}{\emph{colvar}}{}
This function sets the walker at a specific configuration in the collective variable space.
At minimum, any implementation will need to give it the configuration.

\end{fulllineitems}

\index{getColvars() (walker.walker method)}

\begin{fulllineitems}
\phantomsection\label{src/src.doc:walker.walker.getColvars}\pysiglinewithargsret{\bfcode{getColvars}}{}{}
This function returns the location of the walker in the collective variable space.

\end{fulllineitems}

\index{getConfig() (walker.walker method)}

\begin{fulllineitems}
\phantomsection\label{src/src.doc:walker.walker.getConfig}\pysiglinewithargsret{\bfcode{getConfig}}{}{}
This function should return the configuration of the walker in configurtion space.

\end{fulllineitems}

\index{propagate() (walker.walker method)}

\begin{fulllineitems}
\phantomsection\label{src/src.doc:walker.walker.propagate}\pysiglinewithargsret{\bfcode{propagate}}{\emph{numsteps}}{}
This function propagates the simulation forward a given number of steps.
It takes as an argument at least numsteps, the number of steps to progagate forward.

\end{fulllineitems}

\index{setConfig() (walker.walker method)}

\begin{fulllineitems}
\phantomsection\label{src/src.doc:walker.walker.setConfig}\pysiglinewithargsret{\bfcode{setConfig}}{\emph{configuration}}{}
This function should set the system at a specific place in configurtin space.
At minimum, it should take some sort of specification of the configuration.

\end{fulllineitems}


\end{fulllineitems}



\chapter{LAMMPS Walker Module}
\label{src/src.doc:module-lammpsWalker}\label{src/src.doc:lammps-walker-module}\index{lammpsWalker (module)}
This file implements the LAMMPS walker abstraction layer. The core of this idea is that
\index{lammpsWalker (class in lammpsWalker)}

\begin{fulllineitems}
\phantomsection\label{src/src.doc:lammpsWalker.lammpsWalker}\pysiglinewithargsret{\strong{class }\code{lammpsWalker.}\bfcode{lammpsWalker}}{\emph{inputFilename}, \emph{logFilename}, \emph{index=0}, \emph{debug=False}}{}
This class implements the enhanced sampling walker API for the bindings to the LAMMPS package. To check the math formatting, here is an example \$e\textasciicircum{}\{ipi\} = -1\$.

Some usage issues to note:

1) Collective variables (CVs) are defined to the walker by constructing a list of
CVs internally in the walker.colvars object. These CVs list takes the following
format:

{[}''coordinateType'', atomids....{]}

The coordinate type specifies which type of coordinate the CV is 
(i.e. bond, angle, dihedral, etc.). The next items in the list are the atom 
indecies involved in this specific instance of the CV.

The walker will use this list to initialize them to the underlying LAMMPS objects.
\index{close() (lammpsWalker.lammpsWalker method)}

\begin{fulllineitems}
\phantomsection\label{src/src.doc:lammpsWalker.lammpsWalker.close}\pysiglinewithargsret{\bfcode{close}}{}{}
This function closes the LAMMPS object.

\end{fulllineitems}

\index{command() (lammpsWalker.lammpsWalker method)}

\begin{fulllineitems}
\phantomsection\label{src/src.doc:lammpsWalker.lammpsWalker.command}\pysiglinewithargsret{\bfcode{command}}{\emph{command}}{}
This function allows the user to issue a LAMMPS command directly to the
LAMMPS object.

\end{fulllineitems}

\index{destroyColvars() (lammpsWalker.lammpsWalker method)}

\begin{fulllineitems}
\phantomsection\label{src/src.doc:lammpsWalker.lammpsWalker.destroyColvars}\pysiglinewithargsret{\bfcode{destroyColvars}}{}{}
This function removes the colvars set by initColVars().

\end{fulllineitems}

\index{drawVel() (lammpsWalker.lammpsWalker method)}

\begin{fulllineitems}
\phantomsection\label{src/src.doc:lammpsWalker.lammpsWalker.drawVel}\pysiglinewithargsret{\bfcode{drawVel}}{\emph{distType='gaussian'}, \emph{temperature=310.0}}{}
This function redraws the velocities from a maxwell-boltzmann dist.

\end{fulllineitems}

\index{equilibrate() (lammpsWalker.lammpsWalker method)}

\begin{fulllineitems}
\phantomsection\label{src/src.doc:lammpsWalker.lammpsWalker.equilibrate}\pysiglinewithargsret{\bfcode{equilibrate}}{\emph{center}, \emph{restraint}, \emph{numSteps}}{}
This function prepares a LAMMPS image to be at the specified target 
position given by the vector `center' passed and an arguments.

\end{fulllineitems}

\index{getColvars() (lammpsWalker.lammpsWalker method)}

\begin{fulllineitems}
\phantomsection\label{src/src.doc:lammpsWalker.lammpsWalker.getColvars}\pysiglinewithargsret{\bfcode{getColvars}}{}{}
This function returns the current position of the LAMMPS simulation in 
colvars space.

\end{fulllineitems}

\index{getConfig() (lammpsWalker.lammpsWalker method)}

\begin{fulllineitems}
\phantomsection\label{src/src.doc:lammpsWalker.lammpsWalker.getConfig}\pysiglinewithargsret{\bfcode{getConfig}}{}{}
This function returns the current position of the LAMMPS simulation.

\end{fulllineitems}

\index{getVel() (lammpsWalker.lammpsWalker method)}

\begin{fulllineitems}
\phantomsection\label{src/src.doc:lammpsWalker.lammpsWalker.getVel}\pysiglinewithargsret{\bfcode{getVel}}{}{}
This function returns the current velocities from the LAMMPS simulation.

\end{fulllineitems}

\index{minimize() (lammpsWalker.lammpsWalker method)}

\begin{fulllineitems}
\phantomsection\label{src/src.doc:lammpsWalker.lammpsWalker.minimize}\pysiglinewithargsret{\bfcode{minimize}}{\emph{args=None}}{}
This function runs a minimization routine with the specified type.

\end{fulllineitems}

\index{propagate() (lammpsWalker.lammpsWalker method)}

\begin{fulllineitems}
\phantomsection\label{src/src.doc:lammpsWalker.lammpsWalker.propagate}\pysiglinewithargsret{\bfcode{propagate}}{\emph{numSteps}, \emph{pre='no'}, \emph{post='no'}}{}
This function issues a run command to the underlying dynamics to propagate
the dynamics a given number of steps.

\end{fulllineitems}

\index{reverseVel() (lammpsWalker.lammpsWalker method)}

\begin{fulllineitems}
\phantomsection\label{src/src.doc:lammpsWalker.lammpsWalker.reverseVel}\pysiglinewithargsret{\bfcode{reverseVel}}{}{}
This function reverses the velocities of a given LAMMPS simulation

\end{fulllineitems}

\index{setColvars() (lammpsWalker.lammpsWalker method)}

\begin{fulllineitems}
\phantomsection\label{src/src.doc:lammpsWalker.lammpsWalker.setColvars}\pysiglinewithargsret{\bfcode{setColvars}}{}{}
This function initializes the collective variable for a LAMMPS simulation that is handed to this object.

Currently supports the following cv's:
\begin{itemize}
\item {} 
bond

\item {} 
angle

\item {} 
dihedral

\item {} 
x, y or z position coordinates

\item {} 
x, y or z velocity components

\end{itemize}

These are parsed and sent to the underlying LAMMPS object directly using the LAMMPS syntax for these variable.

The implementation first creates a labeled group in LAMMPS containing the atoms used in the CV. Then a compute is initialized using that group.

\end{fulllineitems}

\index{setConfig() (lammpsWalker.lammpsWalker method)}

\begin{fulllineitems}
\phantomsection\label{src/src.doc:lammpsWalker.lammpsWalker.setConfig}\pysiglinewithargsret{\bfcode{setConfig}}{\emph{config}}{}
This routine sets the internal configuration.

\end{fulllineitems}

\index{setDynamics() (lammpsWalker.lammpsWalker method)}

\begin{fulllineitems}
\phantomsection\label{src/src.doc:lammpsWalker.lammpsWalker.setDynamics}\pysiglinewithargsret{\bfcode{setDynamics}}{}{}
This routine sets the dynamics for the walker.

\end{fulllineitems}

\index{setTemperature() (lammpsWalker.lammpsWalker method)}

\begin{fulllineitems}
\phantomsection\label{src/src.doc:lammpsWalker.lammpsWalker.setTemperature}\pysiglinewithargsret{\bfcode{setTemperature}}{\emph{temp}}{}
This function sets the temperature of the walker object.

NOTE THAT THIS DOES NOT ALTER THE DYNAMICS THERMOSTAT. LAMMPS REQUIRES
RESETING THIS THERMOSTAT. WE WILL LOOK INTO HOW TO DO THIS.

\end{fulllineitems}

\index{setTimestep() (lammpsWalker.lammpsWalker method)}

\begin{fulllineitems}
\phantomsection\label{src/src.doc:lammpsWalker.lammpsWalker.setTimestep}\pysiglinewithargsret{\bfcode{setTimestep}}{\emph{timestep}}{}
This routine sets the dynamics time step.

\end{fulllineitems}

\index{setVel() (lammpsWalker.lammpsWalker method)}

\begin{fulllineitems}
\phantomsection\label{src/src.doc:lammpsWalker.lammpsWalker.setVel}\pysiglinewithargsret{\bfcode{setVel}}{\emph{vel}}{}
This function sets the velocity to the lammps simulation.

\end{fulllineitems}


\end{fulllineitems}



\chapter{Indices and tables}
\label{index:indices-and-tables}\begin{itemize}
\item {} 
\emph{genindex}

\item {} 
\emph{modindex}

\item {} 
\emph{search}

\end{itemize}


\renewcommand{\indexname}{Python Module Index}
\begin{theindex}
\def\bigletter#1{{\Large\sffamily#1}\nopagebreak\vspace{1mm}}
\bigletter{l}
\item {\texttt{lammpsWalker}}, \pageref{src/src.doc:module-lammpsWalker}
\indexspace
\bigletter{w}
\item {\texttt{walker}}, \pageref{src/src.doc:module-walker}
\end{theindex}

\renewcommand{\indexname}{Index}
\printindex
\end{document}
